It has long been a dream to communicate with a computer as one might with another human being using natural language speech and text.
We are now coming closer to this dream, as natural language interfaces become increasingly popular.
Our phones are already reasonably good at recognizing speech, and personal assistants, such as Apple Siri, Google Now, Microsoft Cortana, Amazon Alexa, \textit{etc.}, help us with everyday tasks and answer some of our questions.
Chat bots are arguably considered ``the next big thing'', and a number of startups developing this kind of technology has emerged in Silicon Valley and around the world\footnote{http://time.com/4194063/chatbots-facebook-messenger-kik-wechat/}.

Question answering is one of the major components of such personal assistants.
Existing techniques already allow users to get direct answers to their factoid questions.
%\footnote{https://www.stonetemple.com/rich-answers-in-search/}.
However, there is still a large number of more complex questions, such as advice or accepted general opinions, for which users have to dig into the ``10 blue links'' and extract or synthesize answers from information buried within the retrieved documents.
To cater to these informational needs, community question answering (CQA) sites emerged, such as Yahoo! Answers and Stack Exchange.
These sites provide a popular way to connect information seekers with answerers.
Unfortunately, it can take minutes or hours, and sometimes days, for the community to respond, and some questions are left unanswered altogether. 

To facilitate research on this challenge, TREC LiveQA shared task\footnote{http://www.trec-liveqa.org} was started in 2015, where automated systems attempt to answer real users' questions within a 1 minute period.
This task was successful, with the winning system able to automatically return a reasonable answer to more than half of the submitted questions, as assessed for TREC by the trained judges from NIST.
Nevertheless, many questions were not answered well by any of the participating systems.

In this work we explore two ways \textit{crowdsourcing} can be used to help an automated system answer complex user questions.
The main challenge is how to adapt existing ``batch-mode'' crowdsourcing platforms such as Amazon Mechanical Turk to real-time settings, \textit{e.g.}, to produce an answer within a minute.
More specifically, we study if crowd workers can quickly and reliably judge the quality of the proposed answer candidates, and if it is possible to obtain reasonable written answers from the crowd within a limited amount of time.
To explore these research questions, we introduce our CRQA system, which stands for Crowd-powered Real-time Question Answering.
CRQA integrates a crowdsourcing module into an automated question answering system within an overall learning-to-rank framework for selecting answers to complex questions.
We report extensive experiments of stress-testing the CRQA system, by participating in the TREC LiveQA 2016 evaluation challenge, which provided us with a realistic evaluation setup.

Specifically, the contributions of this paper are threefold:
\begin{enumerate}
	\item CRQA, a novel hybrid near real-time question answering system, that uses crowd workers to rate and augment automatically generated candidate answers.
	\item Large scale experiments using CRQA to answer real user questions in a live TREC LiveQA 2016 challenge setting.
	\item Extensive analysis of the system performance, focusing on the contributions of crowd input to the performance of the overall system.
\end{enumerate}

The results and findings of this work can be useful to guide future research on hybrid human-computer question answering, and automated intelligent assistant systems.

\todo[inline]{Not sure where to put research questions. They are not 1-1 with contributions.}
%Our research questions can be stated as:
%\begin{itemize}
%\item RQ1. Can crowdsourcing be used to improve the performance of a near real-time automated question answering system?
%\item RQ2. What is the relative contribution of candidate answer ratings and answers provided by the workers to the overall question answering performance?
%\item RQ3. What are the trade-offs in performance, cost, and scalability of using crowdsourcing for real-time question answering?
%\end{itemize}

